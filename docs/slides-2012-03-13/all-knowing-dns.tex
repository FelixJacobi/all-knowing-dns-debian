% vim:ts=4:sw=4:expandtab
% © 2012 Michael Stapelberg
%
% use xelatex %<
%
\documentclass[xetex,serif,compress]{beamer}
\usepackage{fontspec}
\usepackage{xunicode} % Unicode extras!
\usepackage{xltxtra}  % Fixes
\usepackage{listings}
\setmainfont{Trebuchet MS}
\setmonofont{Inconsolata}
\usetheme{default}

\setbeamertemplate{frametitle}{
    \color{black}
    \vspace*{0.5cm}
    \hspace*{0.25cm}
    \textbf{\insertframetitle}
    \par
}

% Hide the navigation icons at the bottom of the page
\setbeamertemplate{navigation symbols}{}

\begin{document}

% slide with bullet points
\newcommand{\mslide}[2]{
    \begin{frame}{#1}
        \begin{list}{$\bullet$}{\itemsep=2em}
            #2
        \end{list}
    \end{frame}
}

\frame{
\begin{center}
\vspace{0.5cm}
{\huge AllKnowingDNS}\\
\vspace{0.50cm}
{\large Reverse DNS für $2^{64}$ IPv6-Adressen}\\
\vspace{1.5cm}
sECuRE, 2012-03-13\\
\end{center}
}

\begin{frame}[fragile]{Reverse DNS?}
    \begin{list}{$\bullet$}{\itemsep=1em}
        \item Man möchte wissen, welcher Name zu einer IP-Adresse gehört:
    \end{list}

    \begin{verbatim}
$ host 192.168.1.24
24.1.168.192.in-addr.arpa domain name pointer midna.lan.
    \end{verbatim}

    \begin{list}{$\bullet$}{\itemsep=1em}
        \item Bei IPv6 ist jeder Nibble eine Einheit:
    \end{list}

    \begin{verbatim}
$ host 2001:4d88:100e:23:3a60:77ff:feab:d3ea   
a.e.3.d.b.a.e.f.f.f.7.7.0.6.a.3.3.2.0.0.e.0. \
0.1.8.8.d.4.1.0.0.2.ip6.arpa domain name pointer midna.lan.
    \end{verbatim}
\end{frame}

\begin{frame}[fragile]{Server-seitig: Zonefile}
    \begin{lstlisting}[basicstyle=\small\ttfamily]
; Reverse lookup 1.168.192.in-addr.arpa
;ORIGIN TTL             Authoritative Server    Email
$TTL 172800
@       172800  IN SOA  dns.lan.    hostmaster.lan. (
         2010120201    ; Serial (YYYYMMDDnn)
             172800    ; Refresh       48 Stunden
              43200    ; Retry         12 Stunden
            1209600    ; Expire         2 Wochen
             172800 )  ; Minimum TTL    2 Tage
; DNS server
                IN NS   dns.lan.
; PTR records
24              IN PTR  midna.lan.
    \end{lstlisting}
\end{frame}


\mslide{Problem}{
	\item RaumZeitLabor: \texttt{2001:4d88:100e:ccc0::/64}\\
    $\rightarrow$ $2^{48} = 281474976710656$ potenzielle Adressen (autoconf)
    \item selbst bei 1 Byte/Adresse wäre ein Zonefile 256 TiB groß
    \item \textbf{ICH WILL ABER!}
}

\mslide{Lösung}{
	\item DNS-Server, der Antworten aus der Anfrage generiert
    \item Anfrage: \texttt{2001:4d88:100e:ccc0:216:eaff:fecb:826}\\
    Antwort: ipv6-\texttt{0216eafffecb0826}.nutzer.raumzeitlabor.de
    \item Da gibts doch bestimmt schon was?
    \item Aber was ist mit DNS für Hosts wie \texttt{2001:4d88:100e:ccc0:219:dbff:fe43:2ec5} (blackbox.raumzeitlabor.de)?
}

\mslide{AllKnowingDNS}{
	\item \texttt{Net::DNS::Nameserver} + ein paar Zeilen Perl (\textasciitilde{} 400 SLOC)
    \item Binnen zwei Stunden entstanden
    \item Nochmal ein Tag für Konfiguration, Refactoring
}

\begin{frame}[fragile]{AllKnowingDNS: Konfiguration}
    \begin{lstlisting}[basicstyle=\small\ttfamily]
$ cat /etc/all-knowing-dns.conf
# Configuration file for AllKnowingDNS v1.0

# RaumZeitLabor
network 2001:4d88:100e:ccc0::/64
   resolves to ipv6-%DIGITS%.nutzer.raumzeitlabor.de
   with upstream 2001:4d88:100e:1::2

listen 2001:4d88:100e:1::3
listen 79.140.39.197
    \end{lstlisting}
\end{frame}

\begin{frame}[fragile]{AllKnowingDNS: Reverse-Delegation}
    \begin{lstlisting}[basicstyle=\small\ttfamily]
$ORIGIN .
$TTL 604800     ; 1 week
e.0.0.1.8.8.d.4.1.0.0.2.ip6.arpa IN SOA
      infra.in.zekjur.net. hostmaster.zekjur.net. (
                    20         ; serial
                    604800     ; refresh (1 week)
                    86400      ; retry (1 day)
                    2419200    ; expire (4 weeks)
                    604800     ; minimum (1 week)
                    )
            NS      libri.sur5r.net.
            NS      infra.in.zekjur.net.

; net for RaumZeitLabor
0.c.c.c.e.0.0.1.8.8.d.4.1.0.0.2.ip6.arpa.
    IN  NS        ipv6-rdns.zekjur.net.
    \end{lstlisting}
\end{frame}

\begin{frame}[fragile]{AllKnowingDNS: Forward-Delegation}
    \begin{lstlisting}[basicstyle=\small\ttfamily]
$TTL    6h
raumzeitlabor.de IN SOA ns1.jpru.de. hostmaster.jpru.de. (
                        2012030701
                        3h
                        30m
                        7d
                        1d )
    IN  NS  ns1.jpru.de.
    IN  NS  ns2.jpru.de.

nutzer.raumzeitlabor.de. IN NS ipv6-rdns.zekjur.net.
    \end{lstlisting}
\end{frame}

\begin{frame}[fragile]{AllKnowingDNS: Query-Log}
    \begin{lstlisting}[basicstyle=\scriptsize\ttfamily]
2012-03-09 11:40:40 +0100 - 178.7.42.88 - query for \
ipv6-021075fffe1adf8f.nutzer.raumzeitlabor.de (AAAA)

2012-03-09 12:20:40 +0100 - 178.7.42.88 - query for \
f.8.f.d.a.1.e.f.f.f.5.7.0.1.2.0.0.c.c.c.e.0.0.1.8.8.d.4.1.0.0.2.ip6.arpa (PTR)

2012-03-09 12:40:17 +0100 - 188.40.25.2 - query for \
8.d.a.e.0.6.e.f.f.f.4.b.8.8.2.a.0.c.c.c.e.0.0.1.8.8.d.4.1.0.0.2.ip6.arpa (PTR)

2012-03-09 12:40:17 +0100 - 2a01:4f8:0:a111::add:9898 - \
query for ipv6-a288b4fffe60ead8.nutzer.raumzeitlabor.de (AAAA)
    \end{lstlisting}
\end{frame}

\mslide{EOF}{
	\item \url{http://all-knowing-dns.zekjur.net/}
    \item \url{http://code.stapelberg.de/git/all-knowing-dns/}
    \item michael+akd@stapelberg.de
    \item Fragen?
}

\end{document}
